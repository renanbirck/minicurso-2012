\documentclass{beamer}

\usepackage{default}
\usepackage[portuguese]{babel}
\usepackage[utf8]{inputenc}
\usepackage{graphicx}
\usepackage{hyperref}
\usepackage{mdwlist}
\usepackage{textcomp}
\usepackage{graphicx}
\usetheme{Copenhagen}

\title{\textit{LTspice IV} (um pouco mais) avançado}
\author{Renan Birck Pinheiro}
\institute{Universidade Federal de Santa Maria}

\begin{document}

\begin{frame}
\titlepage
\end{frame}

\begin{frame} % slide introdução
\frametitle{Parâmetros}
\begin{itemize}
\item É interessante usar parâmetros quando temos vários componentes com o mesmo valor.
\item Sintaxe: \texttt{.param NOME VALOR}; e daí em diante se referir usando \{NOME\} (assim mesmo, entre colchetes).
\item Também podemos definir valores de componentes como expressões matemáticas usando parâmetros.
\end{itemize}
\end{frame} % slide introdução

\begin{frame} % slide introdução
\frametitle{Varredura de parâmetros}
\begin{itemize}
\item Permite repetir uma simulação variando parâmetros.
\item Sintaxe: \texttt{.step param NOME MIN MAX PASSO}; para variar o paràmetro \texttt{NOME} de \texttt{MIN} a \texttt{MAX} em passos de \texttt{PASSO}.
\end{itemize}
\end{frame} % slide introdução

\begin{frame}
\frametitle{Ex.: Potenciômetro}
\begin{itemize}
\item Um resistor de valor $R$ e outro de valor $R_max - R$.

\end{itemize}
\end{frame}

\begin{frame}
\frametitle{O comando \texttt{.measure}}
\begin{itemize}
\item Permite medições automatizadas, sem precisar ficar lendo gráficos.
\item Sintaxes possíveis:
\begin{itemize}
\item \texttt{.meas TIPO\_ANALISE NOME find VALOR at=TEMPO} (medir VALOR quando atingir TEMPO)
\item \texttt{.meas TIPO\_ANALISE NOME AVG/RMS/MAX/MIN/INTEG/PP expressao} (medir média/RMS/máximo/mínimo/integral/pico-pico da expressão)
\end{itemize}
\item Ex.: \texttt{.meas tran ganho find v(out) at=15m} (medir V(out) em 15 ms)
\item Ex.: \texttt{.meas tran media\_consumo avg v(out)*I(Rload)} (medir média da dissipação na carga)

\end{itemize}
\end{frame}

\begin{frame}
\frametitle{Obtendo modelos dos sites de fabricantes}
\end{frame}

\begin{frame}
\frametitle{Simulação digital}
\begin{itemize}
\item O LTspice tem recursos básicos para a simulação de circuitos digitais. Estão na categoria de componentes "Digital"
\item 
\end{itemize}
\begin{frame}

\begin{frame}
\frametitle{Projeto usando blocos}
\end{frame}

{\LARGE OBRIGADO!}
\end{frame}

\begin{frame}
Contatos: \url{renan.ee.ufsm@gmail.com} \url{http://facebook.com/renanbirck} \newline

O código-fonte desses slides e os circuitos empregados estão disponíveis em \url{https://github.com/renanbirck/minicurso-2012} ou com o autor.
\end{frame}



\end{document}
