\documentclass{beamer}

\usepackage{default}
\usepackage[portuguese]{babel}
\usepackage[utf8]{inputenc}
\usepackage{graphicx}
\usepackage{hyperref}
\usepackage{mdwlist}
\usepackage{textcomp}
\usepackage{graphicx}
\usetheme{Copenhagen}


\title{\textit{LTspice IV} (um pouco mais) avançado}
\author{Renan Birck Pinheiro}
\institute{Universidade Federal de Santa Maria}

\begin{document}

\begin{frame}
\titlepage
\end{frame}

\begin{frame} % slide introdução
\frametitle{Varredura de parâmetros}
\begin{itemize}
\item É interessante usar parâmetros quando temos vários componentes com o mesmo valor.
\item Sintaxe: \texttt{.param NOME VALOR}; e daí em diante se referir usando \{NOME\} (assim mesmo, entre colchetes).
\end{itemize}
\end{frame} % slide introdução

\begin{frame}
\frametitle{O comando \texttt{.measure}}
\end{frame}

\begin{frame}
\frametitle{Obtendo modelos dos sites de fabricantes}
\end{frame}

\begin{frame}
{\LARGE OBRIGADO!}
\end{frame}

\begin{frame}
Contatos: \url{renan.ee.ufsm@gmail.com} \url{http://facebook.com/renanbirck} \newline

O código-fonte desses slides e os circuitos empregados estão disponíveis em \url{https://github.com/renanbirck/minicurso-2012} ou com o autor.

Crédito das tirinhas: Vida de Programador \url{http://www.vidadeprogramador.com.br}
\end{frame}



\end{document}
