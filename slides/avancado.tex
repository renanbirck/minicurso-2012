\documentclass{beamer}

\usepackage{default}
\usepackage[portuguese]{babel}
\usepackage[utf8]{inputenc}
\usepackage{graphicx}
\usepackage{hyperref}
\usepackage{mdwlist}
\usepackage{textcomp}
\usepackage{graphicx}
\usetheme{Copenhagen}

\title{\textit{LTspice IV} (um pouco mais) avançado}
\author{Renan Birck Pinheiro}
\institute{Universidade Federal de Santa Maria}

\begin{document}

\begin{frame}
\titlepage
\end{frame}

\begin{frame} % slide introdução
\frametitle{Parâmetros}
\begin{itemize}
\item É interessante usar parâmetros quando temos vários componentes com o mesmo valor.
\item Sintaxe: \texttt{.param NOME VALOR}; e daí em diante se referir usando \{NOME\} (assim mesmo, entre colchetes).
\item Também podemos definir valores de componentes como expressões matemáticas usando parâmetros.
\end{itemize}
\end{frame} % slide introdução

\begin{frame} % slide introdução
\frametitle{Varredura de parâmetros}
\begin{itemize}
\item Permite repetir uma simulação variando parâmetros.
\item Sintaxe: \texttt{.step param NOME MIN MAX PASSO}; para variar o paràmetro \texttt{NOME} de \texttt{MIN} a \texttt{MAX} em passos de \texttt{PASSO}.
\end{itemize}
\end{frame} % slide introdução

\begin{frame}
\frametitle{Ex.: Potenciômetro}
\begin{itemize}
\item Um resistor de valor $R$ e outro de valor $R_max - R$.
\end{itemize}
\end{frame}

\begin{frame}
\frametitle{O comando \texttt{.measure}}
\begin{itemize}
\item Permite medições automatizadas, sem precisar ficar lendo gráficos.
\item Sintaxes possíveis:
\begin{itemize}
\item \texttt{.meas TIPO\_ANALISE NOME find VALOR at=TEMPO} (medir VALOR quando atingir TEMPO)
\item \texttt{.meas TIPO\_ANALISE NOME AVG/RMS/MAX/MIN/INTEG/PP expressao} (medir média/RMS/máximo/mínimo/integral/pico-pico da expressão)
\end{itemize}
\item Ex.: \texttt{.meas tran ganho find v(out) at=15m} (medir V(out) em 15 ms)
\item Ex.: \texttt{.meas tran media\_consumo avg v(out)*I(Rload)} (medir média da dissipação na carga)
\end{itemize}
\end{frame}


\begin{frame}
\frametitle{Obtendo modelos dos sites de fabricantes}
\end{frame}

\begin{frame}
\frametitle{... não funcionou!}
Pode acontecer do LTspice não conseguir simular um circuito.
\begin{itemize}
\item Verifique se você colocou terra, que todos os componentes estão com valores coerentes (cuidado pra não confundir m e Meg)
\item Coloque uma resistência em série com cada fonte de tensão.
\item Tente usar a opção \texttt{.option cshunt=1e-12} (ela coloca um capacitor em cada nó para tentar melhorar a convergência)
\end{itemize}
Se não adiantar, então um dos modelos que você está usando é de má qualidade (acontece com muitos fabricantes).
\end{frame}


\begin{frame}
\frametitle{Simulação digital}
\begin{itemize}
\item O LTspice tem recursos básicos para a simulação de circuitos digitais. Estão na categoria de componentes "Digital".
\item Parâmetros: Vhigh, Vlow e td (Tensão do \texttt{1}, tensão do \texttt{0} e atraso de propagação)
\end{itemize}
\end{frame}

\begin{frame}
\frametitle{Projetando com blocos}
\begin{itemize}
\item Permite que não precisemos repetir o mesmo circuito várias vezes, basta instanciar um bloco.
\item Blocos podem ser reutilizados entre projetos diferentes, podendo-se criar uma biblioteca de blocos.
\end{itemize}
\end{frame}

\begin{frame}
\frametitle{Criando blocos}
\begin{itemize}
\item Desenhe o circuito a ser transformado em bloco, colocando conectores nas entradas/saídas deles.
\item Podemos usar parâmetros dentro de um subcircuito (usando a sintaxe do \texttt{.param}).
\item Salve esse circuito em um diretório novo (para facilitar a organização).
\item No menu Hierarchy, selecione \texttt{Open this sheet's symbol}. Ele vai pedir para confirmar a criação de um novo símbolo.
\end{itemize}
\end{frame}

\begin{frame}
\frametitle{Usando blocos}
\begin{itemize}
\item Crie um circuito vazio e salve no mesmo diretório onde você salvou o circuito anterior.
\item Adicione um componente normalmente, mas na caixa \textit{Top directory} escolha o diretório atual.
\item Daí podemos escolher o símbolo que acabamos de criar. E para passar parâmetros para ele... clique com o botão direito e preencha a caixa PARAMS.
\end{itemize}
\end{frame}



\begin{frame}
Contatos: \url{renan.ee.ufsm@gmail.com} \url{http://facebook.com/renanbirck} \newline

O código-fonte desses slides e os circuitos empregados estão disponíveis em \url{https://github.com/renanbirck/minicurso-2012} ou com o autor.
\end{frame}


\end{document}
