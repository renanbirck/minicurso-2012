\documentclass[]{report}
\usepackage[portuguese]{babel}
\usepackage[utf8]{inputenc}
\usepackage{graphicx}
\usepackage{hyperref}
\usepackage{mdwlist}

% Title Page
\title{\textbf{Introdução à simulação de circuitos com o \textit{LTspice IV}}}
\author{Renan Birck Pinheiro}



\begin{document}

\begin{titlepage}
\begin{center}

\textsc{\LARGE Universidade Federal de Santa Maria}\\[1.5cm]
\textsc{\Large Centro de Tecnologia}\\[0.5cm]

\end{center}

\vspace*{5cm}
\begin{center}
{\huge \bfseries Simulação de Circuitos com o \textit{LTspice}}\\[0.4cm]
\vspace*{200px}
\emph{Autor:}
Renan Birck Pinheiro [\url{renan.ee.ufsm@gmail.com}]


Santa Maria, \today.
\end{center}




\end{titlepage}


\chapter{Introdução}
Muitas vezes é necessário projetar circuitos analógicos para as mais diversas aplicações.
Neste minicurso, 
\chapter{A interface do LTspice}
Após instalado, o LTspice pode ser aberto a partir do Menu Iniciar ou do ícone criado na área de trabalho.

\chapter{Links úteis e contato}
\item \url{http://tech.groups.yahoo.com/group/LTspice/} - grupo de usuários do LTspice

O autor pode ser contatado atráves de:

\begin{itemize}
\item e-mail/Google Talk: \url{renan.ee.ufsm@gmail.com}
\item Facebook: \url{http://facebook.com/renanbirck}
\item Twitter: \url{http://twitter.com/renan2112}
\end{itemize}
\end{document}          
