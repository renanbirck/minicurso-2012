\documentclass[]{report}
\usepackage[portuguese]{babel}
\usepackage[utf8]{inputenc}
\usepackage{graphicx}
\usepackage{hyperref}
\usepackage{mdwlist}
\usepackage{pslatex}
% Title Page
\title{\textbf{Introdução à simulação de circuitos com o \textit{LTspice IV}}}
\author{Renan Birck Pinheiro}



\begin{document}

\begin{titlepage}
\begin{center}

\textsc{\LARGE Universidade Federal de Santa Maria}\\[1.5cm]
\textsc{\Large Centro de Tecnologia}\\[0.5cm]

\end{center}

\vspace*{5cm}
\begin{center}
{\huge \bfseries Simulação de Circuitos com o \textit{LTspice}}\\[0.4cm]
\vspace*{200px}
\emph{Autor:}
Renan Birck Pinheiro [\url{renan.ee.ufsm@gmail.com}]


Santa Maria, \today.
\end{center}




\end{titlepage}


\chapter{Introdução}
A motivação para esse minicurso foi a necessidade de simulação de circuitos em diversas disciplinas; especificamente, ele foi motivado após vários alunos da disciplina de Circuitos Integrados Analógicos apresentaram dificuldade no uso do \textit{LTspice IV}, necessário à realização das atividades da cadeira.

Aqui serão abordados os conceitos básicos do uso do software. Assuntos mais avançados, que infelizmente não podem ser abordados por questão de tempo, estão disponíveis no sistema de ajuda (aperte F1).

\chapter{Objetivos}
O principal objetivo...

\chapter{A interface do LTspice}
Após instalado, o LTspice pode ser aberto a partir do Menu Iniciar ou do ícone criado na área de trabalho.

\chapter{Links úteis e contato}
\begin{itemize}
\item \url{http://tech.groups.yahoo.com/group/LTspice/} - grupo de usuários do LTspice
\end{itemize}

O autor pode ser contatado atráves de:

\begin{itemize}
\item e-mail/Google Talk: \url{renan.ee.ufsm@gmail.com}
\item Facebook: \url{http://facebook.com/renanbirck}
\item Twitter: \url{http://twitter.com/renan2112}
\end{itemize}
\end{document}          
